\input tex/hdr

{\bf
PF:  I think we need a picture earlier that illustrates the turbulent
length scales and highlights how peta-/exascale will be important.

We should also update the PBR with the 352K picture?

MM: Please feel free to move PBR image to right locagion.
}


%%%%%%%%%%%%%%%%%%%%%%%%%%%%%%%%%%%%%%%%%%%%%%%%%%%%%%%%%%%%%%%%%%%%%%%%%%
\begin{figure}[!ht]
\centering
\includegraphics[width=0.93\textwidth]{figures/350K_peb.png}
 \caption{\label{fig:350k} Reactor simulations of turbulent flow performed
  on OLCF/Summit:
  full core with 352,625 pebbles using an all-hex mesh comprising
  $E$=98,782,067 elements of order $N=8$.}
\end{figure}
%%%%%%%%%%%%%%%%%%%%%%%%%%%%%%%%%%%%%%%%%%%%%%%%%%%%%%%%%%%%%%%%%%%%%%%%%%



Nuclear energy stemming from advanced small modular reactor (SMR) designs
holds promise as a reliable carbon-free resource capable of meeting our
nation's and the world's energy needs.  A wave of investment over the past
several years has spurred innovation in new SMR designs. 
The U.S. is home to over 60 private sector advanced nuclear projects \cite{third_way},
with reactor concepts including a passively-safe Light Water SMR from NuScale \cite{nuscale},
a Salt Pebble Bed Reactor from Kairos \cite{kairos},
and a Sodium Fast Reactor (SFR) coupled with energy storage from Terrapower, 
a company co-founded by Bill Gates \cite{natrium}. Recent growth in nuclear's private
sector represents an important trend as academic research projects mature
toward commercialization.

The design, certification, and licensing of novel reactor designs pose
formidable hurdles to the successful deployment of such technologies. 
The lack of integral-effect test facilities for a wide range of scenarios and
conditions considered in risk-informed analysis leads to a severe deficit of
directly relevant data for these advanced designs.  Development of  appropriate
models for full-system analysis will require {\em high-fidelity numerical
simulations} coupled with advanced instrumented separate-effect experiments to
lay the ground work for subsequent integral-effect tests, when the related
facilities investment is justified.

Our objective is to provide the high-fidelity simulation capabilities that are
essential to this mission.  In particular, we consider SFRs and Light Water SMRs.
Both of these designs have received considerable
attention in recent years. The cores of such reactors comprise tens of
thousands of fueled rods, grouped in bundles of hundreds of rods. Coolant flow,
which is the central energy-transfer mechanism between the fissioning fuel and
electricity-generating turbines, is established between the rods to remove heat
from the nuclear fuel.  Understanding of such fluid flow over a range of
conditions is a major priority and challenge in reactor design and engineering.
Given the scale of the problem (flow in tens of thousands of channels at high
Reynolds number\footnote{The Reynolds number is a non-dimensional measure of flow
speed; high Reynolds number flows are generally turbulent, which makes them
challenging to simulate.}), the
applicability of turbulence-resolving techniques has been limited to small
portions of the reactor core. Significant compromises in accuracy have had to
be accepted in order to perform simulations at the {\em full-core} scale.
These computational economies have implications on the understanding of safety
margins, which ultimately limit economic viability, but also have broader impact
on design constraints that are harder to quantify directly.

Recently, the advent of pre-exascale machines has made possible the simulation
of full reactor cores at moderate Reynolds numbers cite \cite{Fang2021}.  This
proposal seeks to build on these recent achievements to develop a deeper
understanding of core-wide thermal-fluid phenomena.

% I don't like how the footnote interrupts this point. Would like to reformat
\begin{displayquote}
{\it
To accelerate the deployment of SMRs and advanced reactors, this project will
use NekRS, the GPU-oriented version of Nek5000, to establish a full-core Large Eddy
Simulation (LES) database to support development of low-fidelity models.  
The simulations are aimed at providing critical understanding and model
development for core-wide phenomena.  The effort will take place over three
years on the supercomputers Summit and Frontier.\\
\\
\noindent NekRS is particularly performant on current generation GPUs such as the NVIDIA
V100s and A100s and AMD MI100s.  With recent algorithmic developments, it is possible 
to perform a single flow-through time for LES of a full core in just six hours 
using 27684 V100s on Summit.\\
\\
Our prior work in this area has led to deployment of Nek5000/RS within the 
nuclear industry at vendors such as NuScale, Kairos, and Terrapower.
Nek5000 has been approvied by the Nuclear Regulatory Commission (NRC), the
organization tasked with licensing reactors for commercial use,
for purposes in the licensing process for thermal-fluid analysis.
}
\end{displayquote}

\vspace{-.25in} \subsection{The Challenge Problems}
\vspace{-.2in}

Thermal-fluid modeling of nuclear reactors is used to predict the
proximity of the nuclear fuel to various design limits that can affect
personnel radiation dose; the ability of power conversion equipment to reliably
cool the core under a range of normal and degraded/failed equipment conditions;
and the economic viability of new operating paradigms, such as load following.
The accuracy of such simulations has direct implications on design
certification, licensing, and eventual market share. 

Computational Fluid Dynamics (CFD) is used widely in reactor design and safety
analysis, but the scale required for full-core simulation has typically limited
such geometries to less than 1\% of the core -- usually, a single ``fuel
bundle,'' or grouping of rods \cite{wang2020,fanning,wang2020b}. The power and
flow distributions in a reactor can be highly nonuniform, and many physics
phenomena cannot be accurately predicted with single-bundle models -- models that
account for inter-bundle coupling across the entire core are required.  

However, the
scale of full-core CFD has traditionally precluded its use as an analysis tool
in industry. Design teams instead rely on multiscale bridging from single-bundle
experiments/CFD models to low-cost, lower-resolution methods.
These single-bundle experiments/simulations are {\it isolated} from
the full-core physics via insulated or symmetry boundary conditions, and then used
to predict bulk heat and mass transfer corrections in the lower-resolution methods.
  Examples of coarse-mesh methods combined with CFD include the
homogenized porous media method \cite{wang2020c,Kim2020}; the ``subchannel'' method, a
finite volume method specialized to rod-type nuclear fuels \cite{blyth}; and
momentum source models \cite{hu2013}. 

By nature of the scale decoupling between the CFD domain and the full core, a
key outstanding issue with this analysis approach is an inability to reliably 
account for {\it interaction between global and local scales}. This recognized
limitation increases reliance on approximate methods, which in turn further
constrains the reactor design.

Light Water SMRs and SFRs both exhibit significant full-core thermal-fluid
physics for which the available low-resolution analysis methods cannot capture
important local/global interactions. In Light Water SMRs, inter-bundle mixing
due to variations in mass flow rate plays an important role in fluid structure
interaction (FSI) and deposition of coolant contaminants on heat transfer
surfaces ({\bf TODO: references}). Thanks to the rapid 
development of cutting-edge supercomputers, the pin-resolved CFD investigation
for an entire SMR core has been recently achieved by our team \cite{Fang2021}. 
A set of momentum sources was developed to account for the effects of mixing 
spacer grids while the RANS approach was applied to model the turbulence. Further
increase of the simulation fidelities (e.g., LES) will produce high-quality flow databases
much needed to understand/model the SMR thermal-fluid performance at the full
core scale.  % This is good, but I think it should be moved somewhere else
{\bf more discussion about business impact of what we can learn in this proposal}

In SFRs, a critical reactor design consideration is the structural expansion of
the solid fuel components in response to differential temperature and
irradiation damage gradients. The ducted fuel bundles bend and deform,
contacting neighboring bundles and transferring loads across the entire core to
restraint rings welded to the reactor vessel. These changes in core geometry
play an important role in reactor criticality (passive control) and mechanical
refueling operations.  An incomplete understanding of the structural
deformation of SFRs, especially the thermal-flow physics, has resulted in fuel
melting \cite{brittan}, rapid surges in power \cite{chaumont}, and difficulty
refueling \cite{shields} in several SFRs operated around the world.

For neutronic purposes related to transmutation and shielding, neighboring
bundles in SFRs can vary by up to a factor of 100 in power and a factor of 50 in mass
flowrate \cite{abr}. An important driver of structural
expansion in SFRs is heat transfer both within bundles and across thin gaps
between adjacent bundles that contain laminar sodium flow, a heat transfer
mechanism referred to as ``inter-assembly'' heat transfer. Even though
inter-assembly heat transfer is driven by these vastly different flow regimes
(laminar, transitional, and turbulent),
large cross-bundle thermal gradients, and gap sodium flow, industry models for
core thermal-fluids are based on closures obtained from experiments and CFD
models of symmetric, isolated fuel bundles without consideration of 
full-core effects and realistic boundary conditions \cite{touran}. 

Despite the fact that inter-assembly heat transfer is a major contributor 
to the structural behavior of SFR cores, most industry approaches to full-core
analysis neglect the
gap flow entirely \cite{touran}, homogenize the flow into other structural
materials \cite{fiorina_of}, or neglect certain azimuthal heat transfer 
paths between neighboring bundles \cite{touran} -- without correction terms to account for the
un-resolved physics \cite{touran,fiorina_of}. Some researchers substitute Reynolds
Averaged Navier-Stokes (RANS) CFD for small regions in a full-core low-resolution
model \cite{wang2020,gerschenfeld,Kim2020}, an approach that is still subject to
 reduced accuracy compared to resolved CFD simulations. 

The nuclear industry has an acute need for LES-informed models for
inter-assembly heat transfer -- no models exist for inter-assembly heat
transfer considering the coupling between global and local effects. Common
approximations for local heat transfer do not fully characterize this
global heat transfer mechanism, and this knowledge gap
may result in significant
underprediction of heat removal \cite{gerschenfeld}, requiring over-engineered
safety systems or sub-optimal power density. Therefore, the second challenge
problem addressed here is to develop inter-assembly heat transfer models with
full-core LES that properly account for the interaction between flow regime and
power distributions on core-wide fission heat redistribution. 

%{\bf TODO: What other content should go here? Maybe more discussion of typical
%problem sizes people have done before? Typical DOFs for a full-core porous
%media/subchannel solve? Resolution comparison?}

%{\bf Talk about why Petascale resources are required}
%Petascale resources are required to understand how the interactions between
%laminar, mixed convection, and forced convection flow regimes affect
%inter-assembly heat transfer.

%Explain what advances you expect to be enabled by an INCITE award that
%justifies an allocation of petascale resources (e.g., anticipated impact on
%community paradigms, valuable insights into or solving a long-standing
%challenge, etc.). 

% Place the proposed research in the context of competing work in your
% discipline or business. 

% State clearly the challenge problem or problems we are trying to solve.  Make
% clear statements about impact.

\vspace{-.25in}
\subsection{Impact and Insights}
\vspace{-.2in}

This proposal aims to provide insights into two grand challenges in the design
of rod-type nuclear reactors by developing an LES database to support the
development of lower fidelity models. Both grand challenges share the same
motivating gap analysis -- reduced-geometry CFD models cannot properly account
for the interaction between core-wide physics and local thermal-fluids. By
developing more accurate lower fidelity models informed by turbulence-resolved
CFD, this proposal seeks to reduce risk to the technical design, economic
viability, and licensing of advanced reactor concepts. 

The reactor development timeline, from conception to power generation, often
takes decades to accomplish; much of the reactor design and analysis is
``front-loaded'' to allow sufficient time for the licensing process, a
prerequisite to construction. The two challenge problems selected in this
proposal will have immediate technical and business impact to the nuclear
industry by coinciding with large Light Water SMR and SFR development programs
in private industry.

% feel to add some statements that can further highlight the business impact
Several strategically important projects have been identified and are being pursued by 
the Exascale Computing Project (ECP) for the upcoming exascale computing platforms. 
Among these, the ExaSMR project is specifically focused on the SMRs.
SMRs offer the prospect of affordable baseload electricity production while 
avoiding some of the traditional limitations that encumber large reactor designs, 
such as high capital costs and long construction timelines. 
As a result, government, industry, and academia have all expressed interest in 
SMR research and development.
The objective of the ExaSMR project is to use exascale computing platforms to 
carry out extreme-fidelity multi-physics simulations of SMRs, namely, 
to enable verification and validation of current analysis methods by 
generating detailed power distributions during startup conditions, 
full-cycle depletion analysis, and fast transient behavior characteristics of 
accident scenarios \cite{Merzari2018}. 
Significant progress has been made in both the coupling of Monte Carlo (MC) Neutronics 
and CFD \cite{Hamilton2018,Hamilton2019} and the development of CFD modleing 
capability \cite{Fang2021,Merzari2018}.

% maybe emphasize a bit stronger that we're not doing any deformation in this
% proposal here so it's clear?
As part of the Department of Energy (DOE), the Advanced Reactor Demonstration
Program (ARDP) awarded TerraPower, a private nuclear development company
co-founded by Bill Gates, funds
to develop and build an advanced SFR by 2027 \cite{ardp_tp}. The next two to
three years are the opportune moment for the LCF to have an impact on the
viability of this SFR design by making available to industry more accurate
inter-assembly heat transfer models. The thermal-fluid physics is the most important
contributor to transient structural deformation in SFRs \cite{wozniak}, and
structural deformation measurements often have the highest uncertainty among a number of
other important core phenomena \cite{lum}. 

Limited experimental data exists for inter-bundle mixing in Light Water SMRs
and for inter-assembly heat transfer in SFRs, necessitating greater reliance on
modeling and simulation. There
are therefore significant potential business impacts to the success of this
research, such as 1)~reduced calculation uncertainty that enables operation closer
to thermal and irradiation limits; 2)~simplified reactivity control mechanisms;
and 3)~improved predictive maintenance capabilities. 
As the U.S.'s nuclear plant construction
experience has waned over the past decade \cite{schlissel}, the viability of
nuclear power in a clean energy portfolio depends on reducing cost through
uncertainty reductions and design simplifications such as the above.

In addition to direct impacts in the nuclear industry, this research will spur
a paradigm shift in high-resolution modeling of nuclear systems. Competing work
in our field remains directed at the single-bundle level both in CFD
simulations \cite{wang2020,fanning,wang2020b,Boyd2016} and experiments
\cite{Nguyen2017,Bhowmik2021,goth,song_2020,martin2020}. 
Only LCF resources are viable for the proposed full-core LES simulations
-- our simulations will be approximately 1000\(\times\) the size of current
``state-of-the-art'' experiments and CFD simulations, because only by modeling
the entire reactor core can the interaction between global and local
thermal-fluids be properly treated. Our full-core simulations will be
first-of-a-kind in addressing core-wide reactor thermal-fluids and build upon
our group's previous works \cite{Fang2021}. The success of this research will
change the notion
that rod-resolved, full-core CFD is still 5\(+\) years away ({\bf Elia do you
have a good reference for this? I've seen people quote this but I don't recall
where}).

% anticipated impact on community paradigms, valuable insights into or solving
% a long-standing challenge, etc.). 

\vspace{-.25in}
\subsection{Previous INCITE Awards}
\vspace{-.2in}


List any previous INCITE award(s) received and discuss the relationship to the
work proposed. 

\vspace{-.25in}
\section{RESEARCH OBJECTIVE AND MILESTONES} % typically about 6 pages
\vspace{-.2in}

% Describe the proposed research, including its goals and milestones and the
% theoretical and computational methods it employs. Goals and milestones should
% articulate simulation and developmental objectives and be sufficiently
% detailed to assess the progress of the project for each year of any
% allocation granted.  Milestones should correlate with those in Section 4,
% “Milestone Table.” 
% It is especially important that you provide clear connections between the
% project’s overarching milestones, the planned production simulations, and the
% compute time expected to be required for these simulations (e.g., should
% correlate with Section 2.3.i, “Use of Resources Requested”) in the research
% proposal.  You should also make clear any dependencies of milestones on other
% milestones.

In the past ten years, the Nuclear Energy Advanced Modeling and Simulation
(NEAMS) program \cite{sofu2017us} has invested in developing modern advanced
CFD software to facilitate the deployment of advanced reactors. In fact, CFD is
currently in widespread use both in reactor design and in safety analysis, as
testified by the increasing number of articles in the field. Among recent
examples, Roelofs \cite{roelofs2018thermal} illustrates the importance of CFD
to liquid metal reactor design and analysis. Detailed modeling and simulation
is of particular importance for advanced reactors, where simulation can be used
in conjunction with separate effect experiments, in the absence of extensive
integral test data (at least in the initial phases of development).

RANS \cite{conner2010cfd} and occasionally Unsteady Reynolds Averaged
Navier-Stokes (URANS) remain the workhorses for analysis conducted in industry,
research centers, and academia.  However, despite significant advancements in
turbulence modeling in recent years, RANS is limited in terms of accuracy.
Turbulence modeling remains a source of uncertainty in complex engineering
flows, as RANS models are not general and are sensitive to even small geometric
changes \cite{merzari2010numerical}. Moreover, RANS and URANS remain limited in
predicting turbulent fluctuations and high order statistics which may be of
interest in important applications such as flow induced vibrations (FIV) and
FSI \cite{yuan2017flow}.

In contrast to RANS, wall-resolved LES and Direct Numerical Simulation (DNS)
provide a much lesser degree of uncertainty and can provide valuable and
unprecedented insight into the flow physics. Historically, however, these
methods have been limited to small geometries, such as sub-channels
\cite{grotzbach1999direct}, due to the high computational cost associated with
both techniques. From the late 1990s to the end of the 2000s, LES/DNS remained
a niche application for nuclear engineering flows. The review of Grotzbach
\cite{grotzbach1999direct} provides a comprehensive status of the capabilities
available in that time frame. Toward the end of the 2000s however, larger scale
calculations started to emerge \cite{pointer2009simulations}.

Since the advent of petascale computing (i.e, computers
capable of more than 1 petaflop), the simulation of portions of nuclear
components has been demonstrated with LES \cite{merzari2017large}. For example,
the simulation of large portions of single fuel assemblies has been demonstrated,
including conjugate heat transfer calculations \cite{obabko2019}. These
simulations can provide invaluable insight into the flow dynamics, which is
difficult or often impossible to obtain with experiments alone. Moreover, large
LES simulations allow investigation of global effects that otherwise are
impossible to elucidate with smaller portions of the system. NEAMS has invested
considerable resources in scalable high-order CFD software that can leverage
large supercomputers to deliver simulations of unprecedented detail and
scale. Example of some landmark calculations conducted with Nek5000/RS are presented in
Figure~\ref{f:examples}.

\begin{figure}[!ht]
\centering
\includegraphics[width=0.93\textwidth]{../figures/examples.png}
%\includegraphics[width=0.93\textwidth]{./figures/examples.png}
\caption{Examples of large-scale LES/DNS simulation of nuclear reactor flows.
a-b) Cross sections of the flow in helical coil steam generator
\cite{alper2018}; c) Velocity magnitude in a 61-pin wire-wrapped rod bundle
\cite{goth2018comparison}; d) Velocity magnitude in a random pebble bed
\cite{yuan2019}.} \label{f:examples}
\end{figure}

% add more of a description of the proposed research and its goals

\begin{displayquote}
\textit{
This proposal seeks the development of a set of full-core LES
datasets to address knowledge gaps in inter-assembly heat and mass
transfer in nuclear reactors.
}
\end{displayquote}

\vspace{-.25in}
\subsection{Theoretical and Computational Methods}
\vspace{-.2in}

Let us consider first the velocity, continuity, and energy equations that
describe the constant-property incompressible flow of a Newtonian fluid in the
absence of other body or external forces:
\begin{equation}
\frac{\partial  u_i  }{\partial t} +  \frac{\partial}{\partial x_j} \left( u_i u_j \right) =-\frac{1}{\rho} \frac{\partial p}{\partial x_i} + \frac{\partial}{\partial x_j} \left[ \nu \left( \frac{\partial u_i}{\partial x_j} +\frac{\partial u_j}{\partial x_i} \right) \right]
\label{eq:UEqn}
\end{equation}
\begin{equation}
\frac{\partial u_i}{\partial x_i} = 0
\label{eq:rhoEqn}
\end{equation}
\begin{equation}
\rho c_p \left( \frac{\partial T }{\partial t} + u_j \frac{\partial T}{\partial x_j} \right) = \frac{\partial }{\partial x_j} \left( \lambda \frac{\partial T}{\partial x_j} \right)
\label{eq:EEqn}
\end{equation}
where $u$ is the velocity, $p$ is the pressure, $T$ is the temperature, $\rho$
is the  density, $\nu$ is the kinematic viscosity, $\lambda$ is
the thermal conductivity, and $c_p$ is the heat capacity. In natural
convection cases, the Boussinesq or low Mach approximations
\cite{tomboulides1997numerical} are applied, leading to different
formulations available in Nek5000/RS. 
{\it Nek5000/RS is the primary code that will be employed 
in this analysis.}

We restrict the proposed research to turbulence-resolving simulations such as DNS and
wall-resolved LES. In advection-dominated problems such as the
one described by Eqs. \ref{eq:UEqn}--\ref{eq:EEqn} at moderate to high Reynolds numbers,
the high-wavenumber component of transported quantities does not decay
exponentially, as observed in diffusion-dominated problems. Therefore, both
meaningful signals and numerical errors can persist in time, and appropriate
numerical schemes must be selected in order to avoid excessive dispersion.
Nek5000/NekRS, based on the spectral element method, is ideally suited for this
type of analysis. {\bf Any more discussion on this point?}

\vspace{-.25in}
\subsection{Description of Tasks}
\label{sec:tasks}
\vspace{-.2in}

% goals and milestones should articulate simulation and developmental
% objectives and be sufficiently detailed to assess the progress of the project
% for each year

The overarching goals of this proposal are twofold -- first, to develop
reference full-core LES CFD predictions that will be compared against
lower-resolution models to {\it provide unprecedented insight into the interaction
between global and local thermal-fluid effects} for rod bundle nuclear reactors. The second
objective is to {\it develop a full-core LES CFD database to inform low-resolution methods}. Each of these
goals are now detailed in terms of major tasks for each of the challenge
problems.

\vspace{-.25in}
\subsubsection{Light Water Small Modular Reactor (SMR) Challenge Problem}
\vspace{-.2in}
{\bf TODO: describe the prototype that the SMR models will be based on, then
describe the tasks}

\vspace{-.25in}
\subsubsection{Sodium Fast Reactor (SFR) Challenge Problem}
\vspace{-.2in}

To address the SFR challenge problem of inter-assembly heat transfer, full-core
LES CFD simulations will be conducted for the Advanced Burner Reactor (ABR), an
open-source reactor design that is considered a prototype of a large commercial
SFR \cite{abr}. The objectives of these simulations are to 1)~answer
outstanding questions about the nature of inter-assembly flow and heat transfer
and 2)~utilize the produced LES database to generate bulk heat 
transfer correlations to improve the predictive capabilities of
lower-resolution tools. The SFR simulations in this proposal will be directed
in four tasks. Each task will be associated with several milestones, 
described in detail with associated time frames
in Table~\ref{tab:milestones}. The major tasks and relationship
between each task of the 36-month project for the
SFR challenge problem are as follows. 

\vspace{-.15in}
\begin{enumerate}[label=\textbf{\Roman*}]
\item Develop full-core meshes and test at scale to ensure high quality. Mesh convergence studies will be conducted for small groups of bundles under a variety of power and flow gradients to establish applicability to full-core geometries. Develop full-core models based on the ABR specifications \cite{abr}.
\item Develop reference low-resolution full-core models using the NEAMS porous media application, Pronghorn \cite{novak2021b}. These models will be compared against the generated LES database to assess the effect if proper resolution of global/local physics interactions. Low-resolution models will be run with non-LCF computing resources.
\item Conduct full-core LES CFD simulations of the ABR at three power-to-flow ratios\footnote{The ``power-to-flow'' ratio is a measure of the relative balance between core power and core cooling. A power-to-flow ratio of unity indicates nominal operating conditions.} that may be observed during actual operation. These conditions correspond to full power (1000 MWth) and a)~nominal flow (\(Re=90000\)), b)~moderately reduced flow (\(Re=72000\)), and c)~significantly reduced flow (\(Re=60000\)). A fourth condition at decay heat removal conditions (\(Re=650\) and 7.5 MWth) will also be included to cover the very low range in power-to-flow ratio.
\item Compare LES database with reference low-resolution models developed in Task II. Evaluate the physics consequence of industry modeling simplifications to inter-assembly heat transfer.
\item Post-process LES database to generate an effective thermal conductivity model applicable to subchannel and porous media methods. Apply new closure to the models developed in Task II and quantify the improvement in accuracy. 
\end{enumerate}
\vspace{-.15in}

Additional details are now provided for each task. During Task I, full-core meshes will be generated for the prototypic design. Mesh convergence will be performed with \(h\)-type refinement\footnote{$h$-type refinement refers to mesh convergence evaluated by reducing element size.} to leverage past experience with optimal element counts per GPU on Summit. Full-core models will be constructed for the four operating conditions (detailed in Task III) based on openly-available material properties, power, and mass flow distributions. During Task II, equivalent low-resolution models will be constructed and run on workstations available to our group at ANL. 

During Task III, full-core LES CFD simulations will be conducted for four different combinations of core power and flowrate. The first condition corresponds to the nominal reactor state. The second two conditions correspond to power-to-flow ratios 25\% and 50\% higher than nominal conditions, which are indicative of reactor trip points \cite{chaumont} and unmitigated equipment failure, respectively. The last simulation corresponds to decay heat removal after the fission reaction has been terminated. 

In Task IV, the LES database will be compared against the reference low-resolution models to answer several outstanding questions:

\vspace{-.15in}
\begin{itemize}
\item How does inter-assembly heat transfer affect core temperatures? The core outlet temperature distribution has a direct impact on thermal stresses and lifetimes of downstream components.
\item How is inter-assembly heat transfer affected by the power-to-flow ratio? The wide variation in flow regime across the core may result in global heat redistributions with flowrate changes. 
\item Are significant cross-flows generated across the core? In submerged ``pool-type'' reactor concepts, such flows could draw fluid into the core from bypass regions, changing fuel temperature distributions.
\end{itemize}
\vspace{-.15in}

The above questions will be addressed by comparing temperature and mass flux
distributions across the LES database, as well as comparing temperature
distributions between the LES database and lower-resolution models. Finally, in
Task V, the LES database will be used to produce an effective conductivity
model for use in lower-resolution tools. By homogenizing over element sizes
typical of porous media models, the heat flux computed along the boundaries of
the inter-assembly region will be used to compute an effective conductivity
\(\kappa\),
\begin{equation}
q_i^{''}=-\kappa_{ij}\frac{\partial T}{\partial x_j}\ ,
\end{equation}
where \(q^{''}\) is the heat flux and \(\kappa_{ij}\) is a diagonal tensor. In other words, the mechanism by which the LES database will inform lower-resolution tools is by introducing a correction to the heat transfer in the inter-assembly space. Due to the small gap width of the inter-assembly space, Courant-Friedrichs-Lewy (CFL) limits, and ensuing time step limitations, are of paramount important to low-resolution methods. Therefore, low-resolution tools such as those used by Terrapower \cite{touran} approximate the fluid flow in the inter-assembly region as a conducting solid. By supplying a necessary closure, \(\kappa_{ij}\), for this model, the proposed research is {\it compatible with industry modeling tools} and can be rapidly integrated into the ARDP design process.

{\bf Does this research plan sound reasonable?}


\vspace{-.25in}
\subsection{Milestones}
\vspace{-.2in}

% milestones should correlate with those in Section 4, Milestone Table
% provide clear connections between the project's overarching milestones, the planned production simulations, and the compute time expected for these simulations (should correlation with Section 2.3.i - use of resources requested). 

A set of milestones has been developed to spread the computational burden over
the expected 3 year term of the project; these milestones are listed in Table~\ref{tab:milestones} and represent the tasks
in Section~\ref{sec:tasks}. The milestones in Table~\ref{tab:milestones} are directly correlated
with the milestones shown in Section 4, where additional information is provided with respect
to required computing resources.

\begin{table}[h!]
\centering
\caption{Summary of the tasks and milestones associated with this proposal. ``Low-res.'' indicates ``low-resolution.''}
\begin{tabular}{l l c l l}
\toprule
Problem & Task & Milestone & Description & End Date \\
\midrule
\multirow{14}{*}{SFR} & \multirow{3}{*}{I} & A & Mesh convergence studies for clusters of bundles & May 2022 \\
& & B & Complete full-core Nek5000/RS CFD mesh and models & July 2022\\
& & C & Preliminary tests of full-core CFD model & Sep 2022\\\cmidrule{2-5}
& \multirow{3}{*}{II} & A & Complete full-core Pronghorn low-res. models & Nov 2022 \\
& & B & Preliminary tests of full-core low-res. model & Nov 2022\\
& & C & Full-core low-resolution simulations & Dec 2022\\\cmidrule{2-5}
& \multirow{2}{*}{III} & A & Final canonical CFD simulations at \(Re=640\) & Apr 2023\\
& & B & Final canonical CFD simulations at \(Re=6E4\) to \(9E4\) & Dec 2023\\\cmidrule{2-5}
& \multirow{2}{*}{IV} & A & Temperature comparisons between LES and low-res. models & Mar 2024\\
& & B & Crossflow measurements from LES database & Jun 2024\\\cmidrule{2-5}
& \multirow{4}{*}{V} & A & Correlate LES database with effective conductivity model & Sep 2024\\
& & B & Apply conductivity closure to low-res. model & Oct 2024\\
& & C & Repeat full-core low-res. simulations with new closure & Oct 2024\\
& & D & Temperature comparisons between LES and low-res. models & Dec 2024\\
\midrule
\multirow{5}{*}{SMR} & I & 1 &    & Jun  2022 \\
& II & 1 & Final canonical flow simulations         & Sept 2022 \\
& III & \\
& IV & \\
& V & \\
\bottomrule
\end{tabular}
\label{tab:milestones}
\end{table}

For the SFR modeling, Task I encompasses model setup and preliminary testing for
the LES CFD simulations. Our group is experienced at meshing such geometries ({\bf add evidence of this/more detail?}); milestone
I.A will develop guidelines for converged meshes for all subsequent simulations, while milestone
I.B will generate full-core meshes and apply realistic power and mass flow distributions to the model.
Milestone I.C will then conclude Task I by conducting preliminary flow simulations to
determine appropriate time step sizes and flow-through times to establish convergence
of various thermal transport speeds across the coupled core.

During Task II, we will develop low-resolution models of the same reactor system;
due to the small computational requirements of these low-resolution models, mesh
refinement studies will be performed directly on the full core model.

In Task III, we will begin our canonical flow simulations for the \(Re=650\) and then
follow with simulations at higher Reynolds number. This strategy will allow us to
continue gaining insights into the parallelization strategies for the flow case with the
lowest mesh resolution requirements.

{\bf Talk about how we will postprocess the Nek results, and then tie that into IV and V}


\input tex/readiness
