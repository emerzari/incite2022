\documentclass[11pt,letterpaper,english]{article}
\usepackage[T1]{fontenc} % Standard package for selecting font encodings
\usepackage{txfonts} % makes spacing between characters space correctly
\usepackage{xcolor} % Driver-independent color extensions for LaTeX and pdfLaTeX.
%\usepackage{blindtext} % To create text
%\usepackage{mdwlist} % mdwlist for compact enumeration/list items
%\usepackage[pagestyles]{titlesec} % related with sections—namely titles, headers and contents
\usepackage{fancyhdr} % header footer placement

\usepackage[top=1in, bottom=1in, left=1in, right=1in] {geometry} % Margins
\usepackage{graphicx}   % Essential for adding images to you document.

\usepackage{sectsty}
\sectionfont{\large}
\subsectionfont{\normalsize}
\subsubsectionfont{\normalsize \it}

\usepackage{caption}
\captionsetup{labelsep=period}

\pagestyle{fancy} % allows you to use the header and footer commands

\raggedright
\begin{document}

\setlength{\parindent}{0in} % Amount of indentation at the first line of a paragraph.

\pagestyle{fancy} \lhead{Harnessing Exascale: First of a Kind High Fidelity Full Core Simulations} \rhead{E. Merzari et al.} \renewcommand{%
%\pagestyle{fancy} \lhead{Large Eddy Simulation of Rod Bundle Cores} \rhead{E. Merzari et al.} \renewcommand{%
\headrulewidth}{0.0pt}

\centering {\bf Curriculum Vitae}\\
{\bf JUN FANG}\\
{\bf e-mail: fangj@anl.gov, phone: +19195922233} \smallskip

\begin{flushleft} {\bf Professional Preparation}
{\parindent 16pt
   ~\\
   Ph.D.,  2016, North Carolina State University, Nuclear Engineering \\
   M.Eng., 2014, North Carolina State University, Nuclear Engineering \\
   B.Eng., 2012, University of Science and Technology of China, Nuclear Science and Technology \\
}

\vspace{.04in}
{\bf Appointments}
{\parindent 16pt
  ~\\
  2020--Present, Nuclear Engineer, Argonne National Laboratory \\
  2017--2020, Post-doctoral Researcher, Argonne National Laboratory\\
  2016--2017, Post-doctoral Researcher, North Carolina State University\\
}

\vspace{.04in}
{\bf Five Publications Most Relevant to This Proposal}
\vspace{-6pt}
\begin{enumerate} \itemsep1pt \parskip0pt \parsep0pt
\item Fang, J., Shaver, D. R., Min, M., Fischer, P., Lan, Y.-H., Rahaman, R., Romano, P., Benhamadouche, S., Hassan, Y. A., Kraus, A., and Merzari, E. (2021). Feasibility of Full-Core Pin Resolved CFD Simulations of Small Modular Reactor with Momentum Sources. \textit{Nuclear Engineering and Design}, {\bf 378}, 111143.
\item Fang, J., Cambareri, J. J., Li, M., Saini, N., and Bolotnov, I. A. (2020). Interface-Resolved Simulations of Reactor Flows. \textit{Nuclear Technology}, {\bf 206}(2), 133–149. 
\item Cambareri, J. J., Fang, J., and Bolotnov, I. A. (2020). Simulation scaling studies of reactor core two-phase flow using direct numerical simulation. \textit{Nuclear Engineering and Design}, {\bf 358}, 110435. 
\item Fang, J., Cambareri, J. J., Brown, C. S., Feng, J., Gouws, A., Li, M., and Bolotnov, I. A. (2018). Direct numerical simulation of reactor two-phase flows enabled by high-performance computing. \textit{Nuclear Engineering and Design}, {\bf 330}, 409–419.
\item Fang, J., Rasquin, M., and Bolotnov, I. A. (2017). Interface tracking simulations of bubbly flows in PWR relevant geometries. \textit{Nuclear Engineering and Design}, {\bf 312}(Supplement C), 205–213. 
\end{enumerate}

\vspace{-6pt}
{\bf Research Interests and Expertise}
{\parindent 16pt
Computational Fluid Dynamics, Finite Element Analysis, High-Performance Computing, Nuclear Reactor Thermal-hydraulics.
}

\vspace{.04in}
{\bf Synergistic Activities}
\vspace{-6pt}
\begin{enumerate} \itemsep1pt \parskip0pt \parsep0pt
\item Recipient of the Best Paper Award from the 18th International Topical Meeting on Nuclear Reactor Thermal Hydraulics (NURETH-18), 2019, Portland, OR.
\item Co-PI for several ALCC proposals: Multiphase Flow Simulations of Reactor Flows (2020), Toward Full Core Multiphysics High Fidelity Calculations (2020), Multiphase Flow Simulations of Nuclear Reactor Flows (2018).
\item Code developers for massively parallel CFD software, e.g., Nek5000, PHASTA, etc.  
\item Reviewer for Nuclear Engineering and Design, Nuclear Science and Engineeering, Nuclear Technology, etc.
\end{enumerate}

\vspace{-6pt}
{\bf Collaborators ({\emph{past 5 years including name and current institution}})}
{\parindent 16pt
~\\
Igor A. Bolotnov, Cameron S. Brown, Matthew Zimmer, Joseph J. Cambareri; (NC State University); Jinyong Feng (MIT); Dillon Shaver, Aleksandr Obabko, Haomin Yuan, Nadish Saini, Ramesh Balakrishnan, Tim Williams (ANL); Elia Merzari (Penn State University); Jiacai Lu, Gretar Tryggvason (Johns Hopkins University); Kenneth E. Jansen, Riccardo Balin, Ryan Skinner (University of Colorado Boulder); Michel Raquin (Cenaero, Belgium); Cameron Smith (RPI); Ben Matthews (NCAR); Hong Yi (University of North Carolina, Chapel Hill)
}

\end{flushleft}

\end{document}
