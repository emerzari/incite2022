
\documentclass[11pt,letterpaper,english]{article}
\usepackage[T1]{fontenc} % Standard package for selecting font encodings
\usepackage{txfonts} % makes spacing between characters space correctly
\usepackage{xcolor} % Driver-independent color extensions for LaTeX and pdfLaTeX.
%\usepackage{blindtext} % To create text
%\usepackage{mdwlist} % mdwlist for compact enumeration/list items
%\usepackage[pagestyles]{titlesec} % related with sections—namely titles, headers and contents
\usepackage{fancyhdr} % header footer placement

\usepackage[top=1in, bottom=1in, left=1in, right=1in] {geometry} % Margins
\usepackage{graphicx}   % Essential for adding images to you document.

\usepackage{sectsty}
\sectionfont{\large}
\subsectionfont{\normalsize}
\subsubsectionfont{\normalsize \it}

\usepackage{caption}
\captionsetup{labelsep=period}

\pagestyle{fancy} % allows you to use the header and footer commands

\raggedright
\begin{document}

\setlength{\parindent}{0in} % Amount of indentation at the first line of a paragraph.

\pagestyle{fancy} \lhead{Harnessing Exascale: First of a Kind High Fidelity Full Core Simulations} \rhead{E. Merzari et al.} \renewcommand{%
% \pagestyle{fancy} \lhead{Large Eddy Simulation of Rod Bundle Cores} \rhead{E. Merzari et al.} \renewcommand{%
\headrulewidth}{0.0pt}

\centering {\bf Curriculum Vitae }\\
{\bf MISUN MIN}\\
{\bf e-mail: mmin@mcs.anl.gov, phone: +1(630)252-5380} \smallskip

\begin{flushleft} {\bf Professional Preparation}
{\parindent 16pt

Ph.D., 2003, Applied Mathematics, Brown University \\
M.S. 1997, Applied Mathematics, Brown University \\
M.S./B.S. 1993/1991, Mathematics, Hanyang University, South Korea \\
}

\vspace{.04in}
{\bf Appointments}
{\parindent 16pt

2011--present, Computational Scientist, Argonne National Laboratory \\
2006--2011, Assistant Computational Scientist, Argonne National Laboratory\\
2003-2006, Postdoctoral Researcher, Argonne National Laboratory\\
2003, Research Associate, Applied Mathematics, Brown University \\
1998-2002, Research Assistant, Applied Mathematics, Brown University \\
1997, Teaching Assistant, Applied Mathematics, Brown University\\
1994-1996, Mobile Communication System Researcher, Hyundai Electronics, South Korea\\
1991-1993, Teaching Assistant, Mathematics, Hanyang University, South Korea \\
}

\vspace{.04in}
{\bf Five Publications Most Relevant to This Proposal}
\vspace{-6pt}
\begin{enumerate} \itemsep1pt \parskip0pt \parsep0pt
\item Paul Fischer, Misun Min, Thilina Rathnayake, Som Dutta, Tzanio Kolev, Veselin
  Dobrev, Jean-Sylvain Camier, Martin Kronbichler, Tim Warburton, Kasia
  Swirydowicz, and Jed Brown, Scalability of high-performance PDE solvers,
  \textit{International Journal of High Performance Computing Applications}, (in press), 2020.
   https://doi.org/10.1177/1094342020915762\\
\item Martinez Rubio, Javier, Yu-Hsiang Lan, Elia Merzari and Misun Min,
On the use of LES-based turbulent thermal-stress models for rod bundle simulations,
\textit{International Journal of Heat and Mass Transfer},
142, 118399, 2019. \\
\item
Evelyn Otero, Jing Gong, Misun Min, Paul Fischer, Phillip Schlatter, Erwin Laure,
OpenACC acceleration for the Pn-Pn-2 algorithm in Nek5000,
\textit{J. Parallel and Distributed Compting}, vol. 132, pp. 69--78, 2019.  \\
\item Matthew Otten, Jing Gong, Azamat Mametjanov, Aaron Vose, Paul Fischer, and Misun Min,
Hybrid MPI/OpenACC implementation for a high order electromagnetic solver on GPUDirect communication,
\textit{International Journal of High Performance Computing Applications}, Vol. 30, No.3, pp.320-–334, 2016.\\
\item Paul Fischer, Katherine Heisey, and Misun Min, Scaling limits for PDE-based simulation (Invited),
\textit{22nd AIAA Computational Fluid Dynamics Conference, AIAA Aviation} (2015-3049), 2015.
\end{enumerate}

\vspace{-6pt}
{\bf Research Interests and Expertise}
{\parindent 16pt

High-Performance Multiphysics Simulations,
Performance Analysis,
High-Order Spectral Element/Discontinuous Galerkin Methods,
Jacobian-Free Newton Krylov Methods,
Computational Electromagnetics, Computational Fluid Dynamics
}

\vspace{.04in}
{\bf Synergistic Activities}
\vspace{-6pt}
\begin{enumerate} \itemsep1pt \parskip0pt \parsep0pt
\item Lead developer for a high-order spectral-element DG electromagnetics solver, NekCEM.
\item Recipient of 2016 R\&D 100 on "NekCEM/Nek5000: Scalable High-Order Simulation Codes"
\item Argonne PI of ECP Co-Design Center for Efficient Exascale Discretizations (CEED)
\item NSF CFD Review Panel (2019), NSF Numerical PDEs Review Panel (2012);
\item SC18 \& SC16 Technical Paper Review Committee;  ATPESC 2020 Review Committee;
\item Argonne LCRC Allocation Committee
\item Reviewer for Int. J. High Perf. Comp. App.,
J. Optics, J. Comp. Phys.,
J. Sci. Comp., J. Comp. App. Math., App. Numer. Math., SIAM J. Sci. Comp.
\end{enumerate}

\vspace{-6pt}
{\bf Collaborators ({\emph{past 5 years including name and current institution}})}
{\parindent 16pt

Paul Fischer (UIUC/ANL),\\
Tzanio Kolev (LLNL), \\
Ananias Tomboulides (AUTH/ANL),\\
Stefan Kerkemeier (K2/ANL),\\
Stephen Gray (ANL),\\
Tim Warburton (VT), \\
Tzanio Kolev (LLNL), \\
Martin Kronbichler (TUM, Germany), \\
Veselin Dobrev (LLNL), \\
Jed Brown (UC Boulder), \\
Fadil Santosa (U Minnesota), \\
Josh Wilson (U Minnesota), \\
Tony Low (U Minnesota), \\
Michael Sprague (NREL),\\
Shreyas Ananthan (NREL), \\
Matthew Churchfield (NREL), \\
Javier Rubio (Scuderia Toro Rosso),\\
Som Dutta (Utah State),  \\
Jing Gong (KTH/Sweden), \\
Yongjoon Hong (SDSU), \\
YuHsiang Lan (ANL), \\
Saumil Patel (ANL), \\
Taehun Lee (CUNY), \\
Tai-Chia Lin  (NTU/Taiwan), \\
Elia Merzari (PSU/ANL), \\
David Nicholls (UIC), \\
Aleksandr Obabko (ANL), \\
Evelyn Otero (KTH/Sweden), \\
Matthew Otten (ANL). }


\end{flushleft}

\end{document}
