\documentclass[11pt,letterpaper,english]{article}
\usepackage[T1]{fontenc} % Standard package for selecting font encodings
\usepackage{txfonts} % makes spacing between characters space correctly
\usepackage{xcolor} % Driver-independent color extensions for LaTeX and pdfLaTeX.
%\usepackage{blindtext} % To create text
%\usepackage{mdwlist} % mdwlist for compact enumeration/list items
%\usepackage[pagestyles]{titlesec} % related with sections—namely titles, headers and contents
\usepackage{fancyhdr} % header footer placement

\usepackage[top=1in, bottom=1in, left=1in, right=1in] {geometry} % Margins
\usepackage{graphicx}   % Essential for adding images to you document.

\usepackage{sectsty}
\sectionfont{\large}
\subsectionfont{\normalsize}
\subsubsectionfont{\normalsize \it}

\usepackage{caption}
\captionsetup{labelsep=period}

\pagestyle{fancy} % allows you to use the header and footer commands

\raggedright
\begin{document}

\setlength{\parindent}{0in} % Amount of indentation at the first line of a paragraph.

\pagestyle{fancy} \lhead{High-Fidelity Flow Data for Multiscale Bridging} \rhead{E. Merzari et al.} \renewcommand{%
\headrulewidth}{0.0pt}

\centering {\bf Curriculum Vitae}\\
{\bf RUI HU}\\
{\bf e-mail: rhu@anl.gov, phone: +16302521461} \smallskip

\begin{flushleft} {\bf Professional Preparation}
{\parindent 16pt
   ~\\
   Ph.D. 2010, M.S. 2007, Nuclear Engineering, Massachusetts Institute of Technology. \\
   M.S. 2005, B.S. 2002, Nuclear Engineering, Shanghai Jiao Tong University, China. \\
   B.S. 2002, Applied Mathematics, Shanghai Jiao Tong University, China. \\
}

\vspace{.04in}
{\bf Appointments}
{\parindent 16pt
  ~\\
  2019.4-Present: Group Manager for Plant System Analysis, Argonne National Laboratory.\\
  2016.1-Present: Principal Nuclear Engineer, Argonne National Laboratory.\\
  2010.3-2015.12, Postdoc Appointee, Nuclear Engineer, Argonne National Laboratory.\\
  2005.9-2010.2: Research Assistant at Center for Advance Nuclear Systems, MIT.\\
}

\vspace{.04in}
{\bf Five Publications Most Relevant to This Proposal}
\vspace{-6pt}
\begin{enumerate} \itemsep1pt \parskip0pt \parsep0pt
\item Y. Liu, D. Wang, X. Sun, Y. Liu, N. Dinh and R. Hu. Uncertainty quantification for Multiphase-CFD simulations of bubbly flows: a machine learningbased Bayesian approach supported by high-resolution experiments. \textit{Reliability Engineering and Systems Safety}, 2021 (forthcoming).\\
\item Y. Liu, R. Hu and P. Balaprakash. Uncertainty Quantification of Deep Neural Network-Based Turbulence Models for Reactor Transient Analysis. \textit{Proceedings of 2021 ASME Verification and Validation Symposium}, May 19-20, 2021. (accepted) \\
\item Liu, Y., Hu, R., Balaprakash, P., Brunett, A., Obabko, A. Coarse Mesh CFD Turbulence Prediction for Reactor Transient using Densely Connected Convolutional Networks. In \textit{Transactions of the ANS Winter Meeting}, Virtual, November 15- 19, 2020.\\
\item Y. Li, A. Brunett, E. Jennings, H. Abdel-Khalik, T. Mui and R. Hu, ROM-based Surrogate Systems Modeling of EBR-II. \textit{Nuclear Science and Engineering}, 2020. https://doi.org/10.1080/00295639.2020.1840238\\
\item R. Hu, Three-Dimensional Flow Model Development for Thermal Mixing and Stratification Modeling in Reactor System Transient Analyses, \textit{Nuclear Engineering and Design}, 345, 209-215 (2019).
\end{enumerate}

\vspace{-6pt}
{\bf Research Interests and Expertise}
{\parindent 16pt
Nuclear reactor safety, multi-scale thermal fluid modeling methods, multi-scale multi-physics reactor system modeling and simulation, fluid flow and heat transfer, machine learning for nuclear applications.
}

\vspace{.04in}
{\bf Synergistic Activities}
\vspace{-6pt}
\begin{enumerate} \itemsep1pt \parskip0pt \parsep0pt
\item PI for an advanced reactor system analysis tool SAM development
\item Deputy Lead for Thermal Fluids technical area of DOE-NE’s NEAMS program
\item PI for an Argonne LDRD project, \textit{Artificial Intelligence Assisted Safety Modeling and Analysis of Advanced Nuclear Reactors}
\item Co-PI for an NEUP IRP project: \textit{Center of Excellence for Thermal-Fluids Applications in Nuclear Energy: Establishing the knowledgebase for thermal-hydraulic multiscale simulation to accelerate the deployment of advanced reactors}, led by PSU
\item Recipient of 2019 R\&D 100 award, \textit{SAM Reactor System Analysis Code}
\end{enumerate}

\vspace{-6pt}
{\bf Collaborators ({\emph{past 5 years including name and current institution}})}
{\parindent 16pt

Steve Bajorek, Joseph Kelly, Nuclear Regulatory Commission\\
Brandon Haugh, Haihua Zhao, Quan Zhou, Kairos Power LLC\\
Annalisa Manera, Tom Downar, Xiaodong Sun, Won Sik Yang, University of Michigan\\
Raluca Scarlett, University of California Berkeley\\
Yassin Hassan, Texas Agricultural and Mechanical University\\
Cody Permann, David Andrs, Sebastian Schunert, Javi Ortensi, Idaho National Laboratory\\
Bob Salko, Oak Ridge National Laboratory\\
Koroush Shirvan, Massachusetts Institute of Technology\\
Elia Merzari, Penn State University\\
Hany Abdel-Khalik, Purdue University\\
Prasanna Balaprakash, Aleks Obabko, Argonne National Laboratory\\
Yiqi Yu, Tyler Sumner, Matt Bucknor, Dillon Shaver, Argonne National Laboratory\\
Bo Feng, Florent Heidet, Darius Lisowski, Rick Vilim, Argonne National Laboratory.
}
\end{flushleft}

\end{document}
