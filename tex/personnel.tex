\documentclass[11pt,letterpaper,english]{article}
\usepackage[T1]{fontenc} % Standard package for selecting font encodings
\usepackage{txfonts} % makes spacing between characters space correctly
\usepackage{xcolor} % Driver-independent color extensions for LaTeX and pdfLaTeX.
%\usepackage[pagestyles]{titlesec} % related with sections—namely titles, headers and contents
%\usepackage{blindtext} % To create text
%\usepackage{mdwlist} % mdwlist for compact enumeration/list items
\usepackage{fancyhdr} % header footer placement

\usepackage[top=1in, bottom=1in, left=1in, right=1in] {geometry} % Margins
\usepackage{graphicx}  % Essential for adding images to you document.

\usepackage{sectsty}
\sectionfont{\large}
\subsectionfont{\normalsize}
\subsubsectionfont{\normalsize \it}

\usepackage{caption}
\captionsetup{labelsep=period}

\setlength{\parskip}{\baselineskip}%
\setlength{\parindent}{0pt}%

\pagestyle{fancy} % allows you to use the header and footer commands


\raggedright
\begin{document}


\setlength{\parindent}{0in} % Amount of indentation at the first line of a paragraph.

\pagestyle{fancy} \lhead{Harnessing Exascale: First of a Kind High Fidelity Full Core Simulations} \rhead{E. Merzari et al.} \renewcommand{%
%\pagestyle{fancy} \lhead{Large Eddy Simulation of Rod bundle cores} \rhead{E. Merzari et al.} \renewcommand{%
\headrulewidth}{0.0pt}

\begin{center}
\bf {PERSONNEL JUSTIFICATION AND MANAGEMENT PLAN} \\
\end{center}

\vspace{-.25in}
\begin{flushleft}
{\noindent \bf  {Personnel Justification}}

The persons involved in this application have all preassigned roles. The co-PIs responsible for coordinating runs and executing tasks. Elia Merzari will be responsible for the overall  project and the SMR (Small modular Reactor) related tasks. April Novak, Maria Goeppert Mayer fellow at Argonne National Laboratory, will lead the SFR (Sodium Fast Reactors) tasks. In addition to the PI three students are expected to setup numerical models and perform runs (two at Pennsylvania State University and one at University of Illinois Urbana-Champaign) as needed.

\vspace{-.15in}
\begin{itemize}
\item \textbf{Elia Merzari} is the PI and he is responsible for the overall coordination of the project. He has led several ALCC proposals. He will be also responsible for the execution of SMR related tasks.
\item \textbf{Paul Fischer} as chief architect of Nek5000, will participate to ensure that the optimal code options and meshing strategies are employed. Paul Fischer has led numerous INCITE proposals and has over 30 years experience in high performance computing. He is the recipient of a Gordon Bell Award and an R\&D100 Award.
\item \textbf{Misun Min} will coordinate the project requirements with the CEED development team for NekRS and will be responsible for extreme scale test, performance analysis and development.
She is the ANL PI for CEED and the CEED liaison for the ExaSMR project. She is a recipient of 2016 R\&D100 Award for the development of NekCEM/Nek5000 Scalable High-Order Simulation Codes.
\item \textbf{April Novak}, Maria Goeppert Mayer fellow at Argonne National Laboratory, will lead the tasks related to SFRs. April brings a wealth of experience to SFRs to the project, as the developer of Pronghorn and other subchannel codes. She has worked at Terrapower in the past.
\item \textbf{Jun Fang} is a nuclear enginner at Argonne National Laboratory, with extensive experience with high performance computing.  He will support work related to SMRs.
\item \textbf{Student 1 (PSU) - Victor Coppo Leite}, will support development of models and production runs for SFR tasks.
\item \textbf{Student 2 (PSU) - Tri Nguyen}, will support development of models and production runs for SMR tasks.
\item \textbf{Student 3 (UIUC) - TBD}, will support development of models and production runs for SFRs.
\end{itemize}

{\noindent \bf  {Management}}

As stated in the previous section each person in the project has a pre-assigned role. The leadership team comprised of Elia Merzari, Paul Fischer, Misun Min, April Novak and Jun Fang will meet (by teleconference) bi-weekly to review the status of milestones.  Moreover, broader coordination meetings of the center activities will be held twice a year with a broader group of PIs. As PI, Elia Merzari will provides updates on status of the work, such as publications and awards.

Each task has a pre-assigned amount of resources and personnel assigned to them. Students will be supervised by the local PI (Elia Merzari for PSU, Paul Fischer for UIUC).

\end{flushleft}

\end{document}
