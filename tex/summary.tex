\documentclass[11pt,letterpaper,english]{article}
\usepackage[T1]{fontenc} % Standard package for selecting font encodingsamely titles, headers and contents
\usepackage{txfonts} % makes spacing between characters space correctly
\usepackage{xcolor} % Driver-independent color extensions for LaTeX and pdfLaTeX.
%\usepackage[pagestyles,raggedright]{titlesec} % related with sections—n
%\usepackage{blindtext} % To create text
\usepackage{fancyhdr} % header footer placement

\usepackage[top=1in, bottom=1in, left=1in, right=1in] {geometry} % Margins
\usepackage{graphicx}  % Essential for adding images to you document.

\usepackage{sectsty}
\sectionfont{\normalsize}
\subsectionfont{\normalsize}
\subsubsectionfont{\normalsize \it}

\usepackage{caption}
\captionsetup{labelsep=period}


\pagestyle{fancy} % allows you to use the header and footer commands
\usepackage{ragged2e}

\raggedright
\begin{document}

\setlength{\parindent}{0in} % Amount of indentation at the first line of a paragraph.

\pagestyle{fancy} \lhead{Pioneering High-Fidelity Full-Core Simulations at Exascale} \rhead{E. Merzari et al.} \renewcommand{%
%\pagestyle{fancy} \lhead{Harnessing Exascale: First of a Kind High Fidelity Full Core Simulations } \rhead{E. Merzari et al.} \renewcommand{%
%\pagestyle{fancy} \lhead{Large Eddy Simulation of Rod Bundle Cores} \rhead{E. Merzari et al.} \renewcommand{%
\headrulewidth}{0.0pt}

\begin{center}
\bf {PROJECT EXECUTIVE SUMMARY} \\
\end{center}




\bigskip

\textbf{Title}: Pioneering High-Fidelity Full-Core Simulations at Exascale \smallskip
%\textbf{Title}: High-Fidelity Full-Core Simulations \smallskip

\textbf{PI and Co-PI(s)}: Elia Merzari, Paul Fischer, Misun Min, April Novak, Jun Fang \smallskip

\textbf{Applying Institution/Organization}: Pennsylvania State University \smallskip

\textbf{Resource Name(s) and Number of Node Hours Requested}: Summit, 2,584,000 node-hours \smallskip

\textbf{Amount of Storage Requested}: 714 TB (total) \smallskip

\textbf{Executive Summary}: \\
\justify
Advanced nuclear energy holds promise as a reliable, carbon-free energy source capable of
meeting our nation's commitments to addressing climate change. A wave of investment in nuclear
power within the United States and around the world indicates an important maturation of
academic research projects into the commercial space. The design, certification, and licensing
of novel reactor concepts pose formidable hurdles to the successful deployment of new technologies.
The high cost of integral-effect nuclear experiments necessitates
the use of high-fidelity numerical simulations to ensure the viability of nuclear energy
in a clean energy portfolio.\\
\\
The objective of this research is to provide the high-fidelity simulation capabilities essential
to this mission by developing unprecedented insight into core-wide phenomena in rod
bundle nuclear reactors. First of a kind, full-core Large Eddy Simulation (LES) of nuclear reactors will be
conducted on Summit and Frontier.
A high-resolution LES database will be used to inform low-resolution industry analysis
methods for two challenge problems critical to the development of advanced nuclear --
inter-assembly mixing in Light Water Small Modular Reactors (SMRs) and 
inter-assembly heat transfer in Sodium Fast Reactors (SFRs).\\
\\
These physics phenomena
both involve coupling at the full-core level and cannot be understood
with the reduced-scale analyses that have dominated the nuclear space. 
Knowledge gaps in these full-core thermal-fluid phenomena have limited
our understanding and ability to predict component vibrations, fuel melting, and
structural deformation -- and therefore the eventual success of advanced nuclear concepts.
This research presents a compelling science and business impact case
by providing unequalled insight into reactor physics and 
high-resolution datasets to generate lower-resolution closure models
for rapid integration with modeling tools in use by the private nuclear industry.\\
\\
This work will target simulations 1000\(\times\) larger than competing work in our field, and only
with capability computing and petascale-level resources can these insights be gained.
We propose to use NekRS, a GPU-oriented version of the Nek5000 code, which is
a highly-scalable open-source spectral element code for 
Computational Fluid Dynamics (CFD) simulation and a Gordon Bell and R\&D 100
Prize winner. NekRS is uniquely qualified for the present research -- NekRS has
demonstrated scaling on Summit to 27648 GPUs for large nuclear systems and
has undergone extensive performance tuning
 including covering communication with computation, use of mixed precision preconditioners,
 and runtime-optimized communication strategies.\\
 \\
 This research is situated at the opportune moment for LCF to 
 impact the trajectory of advanced nuclear. These first of a kind, full-core LES simulations
 will usher in a new era where such simulations are possible
 and firmly establish the nuclear field as a leader in petascale computing.







\end{document}
